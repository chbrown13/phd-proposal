\section{Tools}

This section outlines concepts for tools to develop and existing systems to observe for evaluating our approaches in making effective recommendations to software engineers.

\subsection{\TOOL}

% Prior work indicates \textit{active help systems} are more effective for providing suggestions to software users than passive help systems, which require users to deliberately seek help~\cite{FischerActiveHelp}.

% Research shows using software engineering tools can improve the quality of software and the efficiency of developers, but in reality developers rarely use them.

To evaluate approaches for making digital nudges to software engineers for development tool adoption, we developed \TOOL. We aim for researchers to be able to extend \TOOL to recommend useful tools to software engineers. Our system is designed to target developer receptivity in recommendations: the \textit{desire} of programmers to produce high-quality code and their \textit{familiarity} with a project's code base. \TOOL~is an automated recommender system designed to suggest software engineering tools to developers on GitHub\footnote{https://github.com}. We target GitHub users because the code hosting and collaboration website has millions of accounts and public repositories, as well as billions of code contributions from developers\footnote{https://octoverse.github.com/}. \TOOL~recommends development tools by integrating with projects' build configuration. With the rise of continuous integration and deployment, many projects implement build systems to automatically compile, test, and release their software more efficiently~\cite{AkondDeployment}. Integrating projects into the build allows developers to easily integrate new tools into their normal software development workflow. We iteratively modified \TOOL to use different recommendation approaches for nudging developers to adopt different software engineering tools. \TOOL uses a human-presenting GitHub account to recommend tools to developers. Prior work found that bots emulating humans are more effective than bot accounts~\cite{AmongTheMachines}, and we also quickly discovered bot accounts are ineffective for recommendations after our original \TOOL~user\footnote{https://github.com/tool-recommender-bot} was flagged and disabled on GitHub for ``opening multiple unsolicited pull requests in other users' repositories" within a few hours of making recommendations at the beginning of our \tele~study. Our goal is for \TOOL~to integrate numerous recommendation approaches and make software engineering tool recommendations using the most effective nudge type(s) when developers are most receptive to adoption.

\subsection{\SUGGS}

GitHub recently introduced a new feature that allows developers to suggest a change to a project's code modified by a user.\footnote{https://help.github.com/articles/incorporating-feedback-in-your-pull-request/\#applying-a-suggested-change} Suggestions can be utilized during pull request reviews, allowing reviewers to propose changes at the exact line of code in question and developers to easily accept or reject the change. Code reviews are another example of a software engineering action shown to improve code quality, and early feedback on the suggestion feature shows they are very popular and widely adopted by GitHub users. We plan to evaluate the effectiveness of GitHub suggestions as examples of a \location. These situated nudges already have over 100,000 uses on GitHub projects and developers are ``quick to adopt suggested changes" and integrate this feature into their code review process.\footnote{https://blog.github.com/2018-11-01-suggested-changes-update/} This research aims to examine situated nudges in the context of \SUGGS~to determine if the location of recommendations impacts the effectiveness of adoption for developers.

\subsection{nudge-bot}
