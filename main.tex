% This is samplepaper.tex, a sample chapter demonstrating the
% LLNCS macro package for Springer Computer Science proceedings;
% Version 2.20 of 2017/10/04
%
\documentclass[runningheads]{llncs}
%
\usepackage{minted}
\usepackage{graphicx}
\usepackage{xspace}
\usepackage{booktabs}

\usepackage{enumitem}
\usepackage{tikz}

\usepackage[para,online,flushleft]{threeparttable}
\usepackage{tabulary}
\usepackage{url}

\graphicspath{{figs/}}
\DeclareGraphicsExtensions{.pdf, .png, .ps}

% Actions
\newcommand{\todo}[1]{{\color{red}\bfseries [[TODO: #1]]}}
\newcommand{\reword}{\todo{RE-WORD!!!}}
\newcommand{\details}{\todo{DETAILS!!!}}
\newlist{todolist}{itemize}{2}
\setlist[todolist]{label=$\square$}
\newcommand{\ser}{R}
\newcommand{\see}{C}

\newcommand\insight[1]{
	\noindent 
	\fcolorbox{gray!20}{gray!20}{
		\parbox{0.92\columnwidth}
		{#1}
		\hspace*{0.5ex}
	}
}

% Tools
\newcommand{\TOOL}{\textsl{nudge-bot}\xspace}
\newcommand{\tool}{\textsl{tool-recommender-bot}\xspace}
\newcommand{\EP}{\textsc{Error Prone}\xspace}
\newcommand{\black}{\textsc{Black}}
\newcommand{\suggtag}{\texttt{```suggestion}\xspace}
\newcommand{\SUGGS}{GitHub \textsl{suggested changes}\xspace}
\newcommand{\pom}{\textit{pom.xml}\xspace}

% Approaches
\newcommand{\tele}{naive \textit{telemarketer design}\xspace}
\newcommand{\process}{digital nudges\xspace}
\newcommand{\location}{\textit{situated} nudges\xspace}
\newcommand{\timing}{\textit{just-in-time} nudges\xspace}

% Short names
\newcommand{\nudge}{\texttt{\TOOL}\xspace}
\newcommand{\nudgeT}{\texttt{\TOOL}}
\newcommand{\proc}{\texttt{\TOOL}\xspace}
\newcommand{\sugg}{\texttt{suggestions}\xspace}
\newcommand{\suggT}{\texttt{suggestions}}
\newcommand{\sorry}{\texttt{sorry}\xspace}
\newcommand{\sorryT}{\texttt{sorry}}
\newcommand{\peer}{\texttt{interactions}\xspace}
\newcommand{\peerT}{\texttt{interactions}}
\newcommand{\loc}{\texttt{situated}\xspace}
\newcommand{\jit}{\texttt{just-in-time}}
\newcommand{\diss}{\texttt{Dissertation}\xspace}
\newcommand{\oral}{\texttt{Oral Prelim Exam}\xspace}




\begin{document}
%
\title{Title}
%
%\titlerunning{Abbreviated paper title}
% If the paper title is too long for the running head, you can set
% an abbreviated paper title here
%
\author{Chris Brown}
%
\authorrunning{Brown}
% First names are abbreviated in the running head.
% If there are more than two authors, 'et al.' is used.
%
\institute{North Carolina State University \\
Raleigh, NC \\
\email{dcbrow10@ncsu.edu} \\
\url{http://www4.ncsu.edu/~dcbrow10/}}
%
\maketitle              % typeset the header of the contribution
%
\begin{abstract}
The abstract should briefly summarize the contents of the paper in
15--250 words.

\keywords{Software Engineering \and Recommender Systems \and Digital Nudge}
\end{abstract}
%
%
% This outline is based on Justin Smith's proposal document
% http://www4.ncsu.edu/~jssmit11/Publications/smith_proposal.pdf
\section{Thesis Statement}

\insight{My very cool statement. Something about digital nudges for developers or invent a new term? This conceptual framework shows that \concept increases developer awareness and usage of software engineering tools...}

\section{Introduction}
\todo{TODO: Use me for marking notes.}

\section{Motivation}
Software engineers don't use tools!

\section{Problem}
Not using tools is bad (wastes time, money, effort, etc.)

\section{Objectives}
Increasing tool awareness can increase developer productivity...

\section{Approach}
Use \concept to make effective recommendations!

\section{Contributions}
List research contributions here:

\section{\concept}
\concept is based on peer interactions~\cite{Murphy-Hill2011PeerInteraction}. Also digital nudging and behavioral psychology.

Brings up an interesting question/theory: Is a developer more likely to accept a recommendation because it is easy/visible + in the context of their work versus other factors, like timeliness, trust (outsider/insider), value, risk, etc.

\section{Proposal Outline}



See Table~\ref{tab1}.
\todo{TODO: Better table~\ref{tab1}}

\begin{table}
\caption{My dissertation proposal}\label{tab1}
\centering
\begin{tabular}{|l|l|l|}
\hline
RQ & Project & Status\\
\hline
1 & Peer Interactions & Done (VL/HCC'17)\\
2 & \TOOL & TODO \\
3 & & TODO \\
\hline
\end{tabular}
\end{table}

\section{Completed Research}

\subsection{``How Software Users Recommend Tools to Each Other" \\(Complete, VL/HCC 2017)}
This project was accepted as a paper and presented at the 2017 Visual Languages and Human-Centric Computing conference~\cite{vlhcc17}. 

\subsubsection{Study Rationale.}

Why did we do this study?

\subsubsection{Research Question.}
This study sought to answer the following question: \\

\textbf{RQ} What characteristics of peer interactions make
recommendations effective?

\subsubsection{Methodology.}

\subsubsection{Results.}

\subsection{\TOOL}

\todo{This will hopefully be submitted before proposal}

\subsubsection{Study Rationale.}

\subsubsection{Research Question.}

\subsubsection{Methodology.}

\subsubsection{Results.}

\section{TODO} 

\todo{TODO: Come up with new project ideas for studies}

\subsubsection{Study Rationale.}

\subsubsection{Research Question.}

\subsubsection{Proposed Study.} 

\subsubsection{Hypothesis.} 

\section{Related Work}

\section{Project Plan}

This section outlines the schedule for completing my dissertation.

\subsection{Project Risk and Mitigation}

\subsection{Thesis Contract}

\begin{itemize}
  \item \textbf{Chapter 1}
  \item \textbf{Chapter 2}
  \item ...
\end{itemize}

\subsection{Proposed Timeline}
Graduate May 2020

\todo{TODO: Fancy figure for timeline}


%
% ---- Bibliography ----
%
% BibTeX users should specify bibliography style 'splncs04'.
% References will then be sorted and formatted in the correct style.
%
% \bibliographystyle{splncs04}
% \bibliography{mybibliography}
%
\bibliographystyle{splncs04}
\bibliography{main}

\end{document}
