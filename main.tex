% This is samplepaper.tex, a sample chapter demonstrating the
% LLNCS macro package for Springer Computer Science proceedings;
% Version 2.20 of 2017/10/04
%
\documentclass[runningheads]{llncs}
%
\usepackage{minted}
\usepackage{graphicx}
\usepackage{xspace}
\usepackage{booktabs}
\usepackage{appendix}
\usepackage{enumitem, amssymb}
\usepackage{tikz}

\usepackage[para,online,flushleft]{threeparttable}
\usepackage{tabulary}
\usepackage{url}

\graphicspath{{figs/}}
\DeclareGraphicsExtensions{.pdf, .png, .ps}

% Actions
\newcommand{\todo}[1]{{\color{red}\bfseries [[TODO: #1]]}}
\newcommand{\reword}{\todo{RE-WORD!!!}}
\newcommand{\details}{\todo{DETAILS!!!}}
\newlist{todolist}{itemize}{2}
\setlist[todolist]{label=$\square$}

\newcommand\insight[1]{
	\noindent 
	\fcolorbox{gray!20}{gray!20}{
		\parbox{0.92\columnwidth}
		{#1}
		\hspace*{0.5ex}
	}
}

% Tools
\newcommand{\TOOL}{\textsl{tool-recommender-bot2}\xspace}
\newcommand{\tool}{\textsl{tool-recommender-bot}\xspace}
\newcommand{\EP}{\textsc{Error Prone}\xspace}
\newcommand{\black}{\textsc{Black}}
\newcommand{\suggtag}{\texttt{```suggestion}\xspace}
\newcommand{\SUGGS}{GitHub \textsl{suggested changes}\xspace}
\newcommand{\pom}{\textit{pom.xml}\xspace}

% Approaches
\newcommand{\tele}{naive \textit{telemarketer design}\xspace}
\newcommand{\process}{\textit{process-appropriate} nudges\xspace}
\newcommand{\location}{\textit{situated} nudges\xspace}
\newcommand{\timing}{\textit{just-in-time} nudges\xspace}

% Short names
\newcommand{\nudge}{\texttt{proposed}}
\newcommand{\proc}{\texttt{process-appropriate}}
\newcommand{\sugg}{\texttt{suggestions}}
\newcommand{\sorry}{\texttt{Sorry to Bother You}}
\newcommand{\peer}{\texttt{peers}}
\newcommand{\loc}{\texttt{situated}}
\newcommand{\jit}{\texttt{just-in-time}}


\begin{document}
%
\title{Title}
%
%\titlerunning{Abbreviated paper title}
% If the paper title is too long for the running head, you can set
% an abbreviated paper title here
%
\author{Chris Brown}
%
\authorrunning{Brown}
% First names are abbreviated in the running head.
% If there are more than two authors, 'et al.' is used.
%
\institute{North Carolina State University \\
Raleigh, NC \\
\email{dcbrow10@ncsu.edu} \\
\url{http://www4.ncsu.edu/~dcbrow10/}}
%
\maketitle              % typeset the header of the contribution
%
\begin{abstract}
The abstract should briefly summarize the contents of the paper in
15--250 words.

\keywords{Software Engineering \and Recommender Systems \and Digital Nudge}
\end{abstract}
% Examples:
% Justin Smith- http://www4.ncsu.edu/~jssmit11/Publications/smith_proposal.pdf
% Titus- http://static.barik.net/barik/proposal/barik_proposal_approved.pdf
\section{Thesis Statement}

\insight{My very cool thesis statement. Something about digital nudges for developers or invent a new term? This conceptual framework shows that \concept increases developer awareness and usage of software engineering tools...}

\section{Introduction}

\subsection{Motivation}
Code quality is very important.

So is software developer productivity

\subsection{Problem}
Code is not high-quality...

Software engineers don't use tools! 

And not using tools is bad (wastes time, money, effort, etc.)~\cite{SoftwareFailWatch}~\cite{NIST}. Studies show debugging costs increase the longer a bug remains in code~\cite{SEEconomics}~\cite{SoftwareAssuranceSDLC}.

\subsection{Objectives}
The objective is to use \concept to improve developer behavior to make them more productive and effective.- tools can increase efficiency in development tasks. 

For example, we developed an Eclipse plugin to improve code navigation called \textsl{Flower}~\cite{Flower}. Our evaluation found that Flower's design principles and implementation improved branchless navigation during code search tasks by an average of 126 seconds.

\subsection{Approach}
Use \concept to make effective recommendations!

\subsection{Contributions}
The research contributions of my dissertation will be:
\begin{enumerate}
  \item a \emph{conceptual framework} \concept
  \item a \emph{set of experiments} to evaluate and provide evidence for this framework (See Table~\ref{tab1})
  \item an \emph{automated recommender system} (\TOOL) that integrates concepts from \concept to make tool recommendations to software engineers
\end{enumerate}

\section{Background}

These terms are key concepts necessary for understanding \concept.

\subsection{Digital Nudge}

\textit{digital nudge}

\subsection{Peer Interaction}

Previous work by Murphy-Hill defines a \textit{peer interaction} as the process of discovering tools from colleagues during normal work activities~\cite{Murphy-Hill2011PeerInteraction}.

\subsection{Tool}

In the context of my research, a \textit{tool} is defined as any software command or feature that accomplishes a task. Similarly, a \textit{software engineering tool} refers to any tool that performs a software development or programming task. Examples include static analysis tools...

\section{Conceptual Framework}
\concept is based on peer interactions~\cite{Murphy-Hill2011PeerInteraction}. Also digital nudging, persuasion/nudge theory, diffusion of innovations, behavioral psychology...

Brings up an interesting question/theory: Is a developer more likely to accept a recommendation because it is easy/visible + in the context of their work versus other factors, like timeliness, trust (outsider/insider), value, risk, etc.


\begin{table}
\caption{My dissertation proposal}\label{tab1}
\centering
\begin{tabular}{|l|l|}
\hline
Project & Status\\
\hline
\texttt{interactions} & Done (VL/HCC'17)\\
\textsl{\texttt{Flower}} \todo{Do 2nd-author papers count?} & Done (VL/HCC'17)\\
\texttt{\TOOL} & TODO \\
\texttt{suggestions} & TODO \\
\hline
\end{tabular}
\end{table}

\section{Experiments and Evaluations}

\subsection{[\texttt{interactions}] ``How Software Users Recommend Tools to Each Other" (Completed, Spring 2017)}

\subsubsection{Study Rationale.}

Prior research suggests peer interactions are the most effective way software developers learn about useful tools compared to other discovery methods~\cite{Murphy-Hill2015HowDoUsers,Murphy-Hill2011PeerInteraction}. The goal of this study was to examine peer interactions in an experimental setting to determine what makes user-to-user recommendations an effective mode of tool discovery and to provide implications for improving automated tool recommendation approaches.

\subsubsection{Research Question.}

\begin{description}
  \item[RQ] What characteristics of peer interactions make recommendations effective?
\end{description}

\subsubsection{Methodology.}

To examine peer interactions between software users, we recruited pairs of participants to observe them complete data analysis tasks for our evaluation. In total we observed 13 pairs of participants, seven pairs of students and six pairs of professional analysts from the NC State Laboratory for Analytic Sciences (LAS).\footnote{https://ncsu-las.org/} The evaluation was divided into two phases with students and LAS employees. Participants were paired together using convenience sampling based on availability. The study tasks data from the Titanic shipwreck based on the data science competition from Kaggle~\cite{KaggleTitanic}. We screen and voice recorded each session to collect data to analyze for our evaluation. Participants were allowed to use the software of their choice to complete the tasks during the study.

To investigate what aspects of peer interactions make tool recommendations effective, we examined five characteristics: Politeness, Persuasiveness, Receptiveness, Time Pressure, and Tool Observability. These characteristics were motivated from previous work in persuasion theory from psychology and interview results from Murphy-Hill's peer interaction studies. Furthermore, we collected additional data for characteristics in Phase 2 including the nature of relationship with their partner (Professional, Personal, Academic, or None) and previous computer-based work with their partner, excluding sending and receiving emails. We used psychology research to compile a list of criteria for politeness~\cite{LeechPolite}, persuasiveness~\cite{ShenPersuasive}, and receptiveness~\cite{FoggPersuasiveTech}. To observe time pressure, we searched for statements made by the participants regarding time during their session. Finally, we examined whether tools recommended between pairs had a graphical user interface to determine tool observability. 

Two researchers watched each session to search for peer interactions. For each recommendation observed, we collected the following: \\- the type of peer interaction (Peer Observation or Peer Recommendation), \\
- the approximate time in the video the recommendation took place, \\
- which participants are the driver and navigator,\\
- the study task,\\
- the method of the driver and navigator (if possible),\\
- the name and type of the recommended feature,\\
- a transcript of the dialogue concerning the new tool,\\
- the reaction of the recommendee,\\
- instances in the study where the tool was re-used,\\
- instances where the tool was ignored for a less efficient method,\\
- the effectiveness, politeness, persuasiveness, and receptiveness scores,\\
- whether the recommendations was under time pressure, and\\
- if the recommendation was discussed during the interview and time of discussion in the video.

\subsubsection{Results.}

This results of this project were published at the 2017 Visual Languages and Human-Centric Computing (VL/HCC) conference~\cite{Interactions}. We observed 142 total tool suggestions between participants in our study, and categorized 71 effective, 35 ineffective, and 36 unknown recommendations. We only found a significant difference in the outcome of recommendations that displayed receptiveness between pairs (Wilcoxon, \textit{p} = 0.0002). The implications of this work indicate that future automated tool recommendation approaches should prioritize user receptiveness in order to make more successful suggestions.

\subsection{[\texttt{bot}] ``\TOOL" (In Progress, Fall 2018)}

\subsubsection{Study Rationale.} 

active help system~\cite{Fischer1984ActiveHelpSystems}

\subsubsection{Research Questions.}

\begin{description}
  \item[RQ1] How applicable is \tool~to real-world software applications?
  \item[RQ2] How useful are recommendations from \tool~to developers?
\end{description}

\subsubsection{Methodology.}

\subsubsection{Results.} This results of this project have been submitted for publication at the 2019 Foundations of Software Engineering (FSE) conference.

\subsection{[\texttt{suggestions}] ``Suggestions Study" (In Progress, Fall 2018)}

\subsubsection{Study Rationale.} Analyze the effectiveness of the new GitHub code suggestion feature for pull requests.\footnote{https://blog.github.com/changelog/2018-10-16-suggested-changes/} Recommendations in a different domain.

\subsubsection{Research Questions.}

\subsubsection{Methodology.}

\subsubsection{Results.} This results of this project have been submitted for publication at the 2019 Foundations of Software Engineering (FSE) conference.

\section{TODO} 

\todo{TODO: Come up with new project ideas for studies}

\subsection{[\texttt{codename}] ``paper title" (Proposed, Fall 2019)}

\subsubsection{Study Rationale.} Something to wrap everything up nicely

\subsubsection{Research Question.}

\subsubsection{Methodology.} 

\subsubsection{Hypothesis.} 

\section{Related Work}

\subsection{Digital Nudge}
stuff about choice architecture, nudge theory, human behavior, psychology, etc.

\subsection{Peer Interaction}

Learning from peers, over-the-shoulder learning, peer observation and recommendation,

\subsection{Software Engineering Tool Adoption}

tools exist, make tasks effective, why don't developers use them?

\subsection{Recommender Systems}

general rec systems (i.e. Netflix), rec systems for SE

\section{Project Plan}

This section outlines the schedule for completing my dissertation.

\subsection{Completed Projects}

\subsection{TODO Projects}

\subsection{Proposed Timeline}
Graduate May 2020


\section{Thesis Contract}

\begin{todolist}
  \item \textbf{Chapter 1}
  \item \textbf{Chapter 2}
  \item ...
\end{todolist}

\section{Appendix}
\appendix
\section{Appendix A}
If necessary...

%
% ---- Bibliography ----
%
% BibTeX users should specify bibliography style 'splncs04'.
% References will then be sorted and formatted in the correct style.
%
% \bibliographystyle{splncs04}
% \bibliography{mybibliography}
%
\bibliographystyle{splncs04}
\bibliography{main}

\end{document}
