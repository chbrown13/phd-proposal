\section{Introduction}

\subsection{Developer Recommendations}

% SE Decisions are important
Humans make approximately 35,000 decisions every day.\footnote{\url{https://go.roberts.edu/leadingedge/the-great-choices-of-strategic-leaders}} Likewise, software engineers are also frequently faced with choices to make while developing code. As our society becomes more and more dependent on software products~\cite{SoftwareEatWorld}, it is becoming increasingly important to find ways to improve the behavior and decision-making of software developers. In his book \textit{The New Kingmakers: How Developers Conquered the World}, Stephen O'Grady describes the influence of software engineers' choices on the economy and society by noting ``Developers are the most-important constituency in technology. They have the power to make or break business, whether by their preferences, their passions, or their own products...Developers are now the real decision makers in technology. Learning how to best negotiate with these New Kingmakers, therefore, could mean the difference between success and failure"~\cite[p.~3-4]{OGrady2013King}. Furthermore, the Hacker Noon blog refers to decision-making as ``the most undervalued skill in software engineering" and ``the most important skill in software development", even moreso than coding skills.\footnote{\url{https://hackernoon.com/decision-making-the-most-undervalued-skill-in-software-engineering-f9b8e5835ca6}} and Li and colleagues discovered that the ability to make effective decisions is an important characteristic of being a great software engineer~\cite{GreatSoftwareEngineer}. 

% Developers need help making decisions
While decision-making is an important aspect of software engineering, developers often make less than ideal choices in their work. For example, software engineers often avoid implementing useful \textit{developer actions}, or processes designed to support programmers in completing software development tasks. For instance, Johnson and colleagues found that developers rarely use static analysis tools to automatically check for defects and prevent errors in code~\cite{Johnson2013Why}. To help developers make better decisions, recommender systems have been created to automatically guide users. The ACM International Conference on Recommender Systems (RecSys) defines recommender systems as ``software applications that aim to support users in their decision-making while interacting with large information spaces"~\footnote{\url{https://recsys.acm.org/}, as quoted by~\cite{RSSE}}. Fischer and colleagues also argue that \textit{active help systems} that can automatically make recommendations to users completing tasks are more effective than \textit{passive help systems} requiring users to seek help~\cite{Fischer1984ActiveHelpSystems}. Similarly, recommendation systems for software engineering (RSSEs) are designed to assist developers in completing various tasks and providing information when making decisions~\cite{RSSE}. For example, Spyglass is an RSSE that suggests code navigation tools in Eclipse to help developers save time and effort searching through code while completing programming tasks~\cite{Spyglass}. 

% Existing recommender systems not effective
While many automated suggestion approaches have been developed to help developers make better choices, research shows that face-to-face recommendations between humans are the most effective. Murphy-Hill and colleagues explored how developers discover new software engineering tools and found that \textit{peer interactions}, or the process of discovering tools from coworkers during normal work activities, is the most effective compared to other technical approaches~\cite{Murphy-Hill2011PeerInteraction}. However, even though face-to-face interactions are the best method for recommendations, they are becoming less practical for recommendations in software engineering. For example, Murphy-Hill also discovered that peer interactions occur infrequently among developers in the workplace~\cite{Murphy-Hill2011PeerInteraction}. There are many barriers to peer interactions in industry, such as mandated tools and processes by management, \textit{physical isolation} where programmers are increasingly working alone remotely, and \textit{developer inertia} where software engineers do not feel the need to share or adopt new practices and tools~\cite{Murphy-Hill2015HowDoUsers}. Additionally, many automated systems have proven ineffective for improving developer decision-making. For example, Viriyakattiyaporn and colleagues found that the inability to deliver suggestions in a timely manner discouraged programmers from adopting recommendations to improve code navigation with Spyglass~\cite{viriyakattiyaporn2009challenges}. Thus, this points to a need for improving RSSEs to improve developer behavior by making effective recommendations.

\subsection{Motivating Example}

To understand the impact of the decline of peer interactions and inadequacy of automated RSSEs on decision-making for software engineers, consider the example of Cassius. Cassius is an experienced software engineer maintaining several popular open source JavaScript projects on GitHub. However, he is unaware of several major bugs that exist in his repositories because he does not implement any static analysis tools in his projects. This is primarily because he is not familiar with useful tools to help developers automatically find and prevent defects in JavaScript code. Additionally, he feels he does not need to use these tools because he has never used them in the past and doesn't want to go through the hassle of integrating new systems into his workflow and development environment. Cassius normally works from home remotely, so he does not have many opportunities to learn about useful tools and processes during face-to-face interactions with peers. Cassius also frequently visits online programming communities and blogs such as HackerNews\footnote{\url{https://news.ycombinator.com/}} and Dev.to\footnote{\url{https://dev.to/}}. However, even though he normally notices articles and posts about new tools while reading through developer blogs, this usually occurs while he is not working and outside of his development environment so he is not motivated and usually forgets to try integrating any of these new tools into his code.

Automated recommendations for tools have also been ineffective in persuading Cassius to adopt new tools and practices. He often receives automated emails suggesting new tools, but usually ignores the these messages as marketing and spam. He also frequently observes pop-ups and tool tips recommending useful tools and features in his integrated development environment (IDE), but Cassius usually disregards those as well. One day, Cassius notices a new pull request on one of his repositories. The pull request introduces Cassius to ESLint\footnote{\url{https://eslint.org/}}, an open source static analysis tool for finding errors in Python code. After further inspection of the PR, he sees it was open by a bot attempting to increase awareness of ESLint and automatically adds it to the build configuration files. He just needs to merge the pull requests to add the static analysis tool to his repositories. However, the pull request changes do not adhere to the style guidelines set or sign the Contributor License Agreement required to submit a change. Additionally, the PR fails continuous integration checks in the build for his project. To avoid extra work understanding and fixing each of the issues, he promptly closes the PR without merging the tool. Later, he notices the same pull request was opened by the bot on several of his repositories. This frustrates Cassius and tarnishes his reputation of ESLint as well as code analysis tools in general, and he angrily closes all the recommendation pull requests. This shows that while using bots as an active help systems is useful for automating tasks such as updating build configuration files and scaling recommendations to a large number of projects and users, they can often be inconvenient and problematic for developers.

\subsection{Research Overview}

My research goal is, when given a developer who is unaware of a useful developer action in a development situation, to identify the most effective strategy to convince them to adopt the action. This goal can also be summed up in the following research question posed by Greg Wilson, software engineering researcher and co-founder of Software Carpentry,\footnote{\url{https://software-carpentry.org/}} who tweeted: \\

\begin{blockquote}
``I think the most interesting topic for software engineering research in the next ten years is, \textit{`How do we get working programmers to actually adopt better practices?'}".\footnote{\url{https://twitter.com/gvwilson/status/1142245508464795649?s=20}}
\end{blockquote} \\

To encourage developers to adopt better practices, we will analyze developer action adoption through the lens of behavioral science and economics. Behavioral science research has examined how to encourage humans to make better decisions, embrace beneficial behaviors, and accept new ideas. One framework used to influence human behavior and decision-making is \textit{nudge theory}. A \textit{nudge} refers to any factor that impacts how people make a decision without providing incentives or banning alternatives~\cite{sunstein2008nudge}. We hypothesize that using automated suggestions to nudge developers about useful software engineering practices can increase adoption and encourage usage of developer actions. 

In this thesis, I will argue that we can improve adoption of useful developer actions by designing recommender systems with concepts from nudge theory. My research aims to implement recommendations for developer actions as digital nudges to software engineers to encourage usage of development tools and processes. To study the impact of nudge theory in developer action recommendations, I will adhere to the following plan of work: 1) determine what makes recommendations effective to developers, 2) examine existing tools for recommending developer actions, and 3) create new automated systems and strategies to improve the effectiveness of recommendations to developers.

\subsection{Research Contributions}

The expected contributions of the research for this thesis include:

\begin{itemize}
    \item a \textit{conceptual framework} that presents new digital nudge types to modify software engineer behavior,
    \item a set of \textit{experiments} to evaluate and provide evidence for the digital nudge framework, and
    \item \TOOL, an \textit{automated recommender system} for recommending software engineering actions to developers with our digital nudge types
\end{itemize}

