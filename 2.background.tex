\section{Background}

This section presents background information on key concepts that are important for my research work.

\subsection{Nudge Theory}

% What is nudge theory and why use it?
To encourage adoption of developer actions, we plan to incorporate concepts from \textit{nudge theory}. A nudge is defined as ``any aspect of choice architecture that alters behavior in a predictable way without forbidding alternatives or significantly changing economic incentives"~\cite[p.~6]{sunstein2008nudge}. Nudges impact how humans make common everyday decisions, such as encouraging people to recycle more by increasing the size of recycling bins\footnote{\url{http://nudges.org/2011/05/02/a-strategy-for-recycling-change-the-recyling-bin-to-garbage-bin-ratio/}} and re-labelling trash cans as ``landfills"\footnote{\url{http://nudges.org/2010/08/25/its-not-a-garbage-can-its-a-small-landfill/}}. In these cases, people still have the option not to recycle and are not rewarded for recycling, but are still persuaded by the size and naming of bins. Nudges are also used on a much larger scale to impact human behavior and decision-making. For example, the UK government implemented a Behavioural Insights Team\footnote{\url{https://www.gov.uk/government/organisations/behavioural-insights-team}}, also known as the Nudge Unit, to improve behavior and decisions by citizens. An example of a nudge from this team includes encouraging companies to improve the recruitment of women, promotion of female employees, and reduce the gender pay gap.\footnote{\url{https://www.bi.team/wp-content/uploads/2018/06/GEO\_BIT\_INSIGHT\_A4\_WEB.pdf}} Nudge units are becoming more popular around the world and similar teams have been implemented in other countries such as the United States, Denmark, and Italy~\cite{DelBalzoNudging}. Nudges are useful for improving human behavior because of their ability to influence the context and environment surrounding decision-making, or \textit{choice architecture}~\cite{thaler2014choice}. Thaler and Sunstein note ``nudges are everywhere" and ``choice architecture, both good and bad, is pervasive and unavoidable...Choice architects can preserve freedom of choice while also nudging people in directions that will improve their lives"~\cite[p.~255]{sunstein2008nudge}. In this work, I plan to study the behavioral impact of integrating nudge theory into recommendations to software developers.

\subsubsection{Digital Nudges}

Furthermore, the use of technology and user-interface design elements to nudge user behavior in digital choice environments is referred to as \textit{digital nudging}~\cite{weinmann2016digitalnudging}. For example, the FitBit\footnote{\url{https://www.fitbit.com/}} smart watch nudges users to increase physical activity and adopt healthier lifestyle behaviors through a digital activity tracker~\cite{weinmann2016digitalnudging}. Digital nudges are becoming more and more important for improving human behavior and decision-making. As more and more decisions are being made online and through technological interfaces in digital choice environments, Weinmann argues understanding digital nudges is becoming increasingly important because the designs of these systems will ``always (either deliberately or accidentally) influences people's choices"~\cite[p.~433]{weinmann2016digitalnudging}. Additionally, Mirsch and colleagues argue that implementing digital nudging provides more advantages because they are ``easier, faster and cheaper" and provide a lot more specified functionality compared to traditional nudges~\cite[p.~635]{mirsch2017digital}.

While most prior work on digital nudges has examined their impact on decision-making for software users, there is limited work exploring how they influence the choices and behavior of software developers. Software engineers are constantly presented with decisions to make in digital choice environments while writing code, such as whether or not utilize programming tools and practices in their work. Nudges are useful for automated recommendations because they are interventions that are ``easy and cheap"~\cite[p.~6]{sunstein2008nudge}. Nudging humans is easy because it does not ban alternative options for choosers, and it's cheap since incentives are not provided to encourage the selection of a target option. My work aims to incorporate these concepts into RSSEs by not inhibiting or providing incentives for developers to restrict their options or force them to select specific processes and tools. In turn, we plan to discover if digitally nudging developers to apply useful software engineering practices rather than ignore them can help persuade developers to adopt better practices when faced with real-world programming tasks.

\subsection{Developer Actions}

Developer actions refer to practices designed to support and aid software developers in the completion of programming tasks. Software engineering researchers have created and evaluated a wide variety of useful developer actions to discover their impact on programming teams and products. For example, research shows activities such as pair programming~\cite{WilliamsPairProgramming} and code reviews~\cite{bacchelli2013codereview} are beneficial practices for software development. Another example of a useful developer action is tool adoption. The IEEE Software Engineering Body of Knowledge (SWEBOK), a suite of widely accepted software engineering practices and standards, suggests using development tools is a ``good practice" and can ``enhance the chances of success over a wide range of project”~\cite[p.~A-4]{SWEBOK}. Additionally, Jazayeri further argues that adopting development tools has become so vital to the software development process that tool usage and the ability to switch between tools should be integrated into software engineering education~\cite{jazayeri2004education}. Software engineering researchers and toolsmiths have created many different development tools to help programmers save time and effort in completing development tasks. For example, static analysis tools are systems which automatically examine code to find and detect errors without running the program. Studies show that static analysis tools are useful for providing benefits to projects such as improving code quality~\cite{GoogleFixit}, preventing errors~\cite{bessey2010few}, decreasing debugging time~\cite{Williams2007FaultFixTime}, lowering development costs, and reducing developer effort~\cite{singh2017staticreview}.

However, despite the fact that research provides evidence applying software engineering tools is useful and worthwhile, developers often ignore them in practice. Researchers have further explored reasons why software engineers avoid using tools for static analysis~\cite{Johnson2013Why}, security~\cite{Xiao2014Security},  debugging~\cite{Cao2010Debugging}, refactoring~\cite{Murphy-HillBarriersRefactoring}, documentation~\cite{Forward2002Documentation}, open source contributing~\cite{mendez2017open}, build automation~\cite{Akond2017BuildTools}, continuous integration~\cite{hilton2017CI}, and more. These tools provide a variety of useful functionality, however software developers often have limited knowledge of these tools and rarely use them in their work. Furthermore, research has explored challenges and barriers for adopting other useful programming activities such as agile processes~\cite{nerur2005challenges}, pull-based development~\cite{gousios2015work}, code reviews~\cite{bacchelli2013codereview}, and more. The goal of my work is to improve recommendations to software engineers to help software engineers overcome these barriers and improve this developer action adoption problem. Tilley and colleagues argue that adoption of research-off-the-shelf (ROTS) software developed for industry practitioners should be a primary goal for software engineering researchers~\cite{Tilley2003ROTS}. Furthermore, improving adoption of useful developer actions can affect the software engineering industry which saw over 3.7 billion users impacted by buggy code and approximately \$1.7 trillion spent on software defects in 2017~\cite{SoftwareFailWatch}. Thus, I aim to explore the impact of using nudge theory to increase awareness and encourage adoption of useful developer actions that can help improve code quality and developer productivity in industry.

 

 %While the design and implementation of software engineering tools and company policies can be barriers preventing developer adoption, this research focuses on the discoverability barrier, where users are unaware that a useful tool exists~\cite{Murphy-HillScreencastingDiscovery}. 

