\section{Background}

This section provides background information on two concepts, \textit{nudge theory} and \textit{developer behaviors}, that are key for the research presented in this proposal.

\subsection{Nudge Theory}

% What is nudge theory and why use it?
To encourage adoption of developer behaviors, we plan to incorporate concepts from \textit{nudge theory}. A nudge is defined as any factor ``that alters behavior in a predictable way without forbidding alternatives or significantly changing economic incentives"~\cite[p.~6]{sunstein2008nudge}. Nudges impact how humans make common everyday decisions, such as encouraging people to recycle more by increasing the size of recycling bins\footnote{\url{http://nudges.org/2011/05/02/a-strategy-for-recycling-change-the-recyling-bin-to-garbage-bin-ratio/}} and re-labelling trash cans as ``landfills"\footnote{\url{http://nudges.org/2010/08/25/its-not-a-garbage-can-its-a-small-landfill/}}. In these cases, people still have the option not to recycle and are not rewarded for recycling, but are still persuaded by the size and naming of bins. Nudges are also used on a much larger scale to impact human behavior and decision-making. For example, the UK government implemented a Behavioural Insights Team\footnote{\url{https://www.gov.uk/government/organisations/behavioural-insights-team}}, also known as the Nudge Unit, to improve behavior and decisions by citizens. An example of a nudge from this team includes encouraging companies to improve the recruitment of women, promotion of female employees, and reduce the gender pay gap by providing guides and feedback to companies encouraging actions such as including multiple women in the recruitment process, encouraging salary negotiations, introducing programs focused on increasing and fostering diversity, and more.\footnote{\url{https://www.bi.team/blogs/new-for-employers-the-latest-evidence-on-what-works-to-reduce-the-gender-pay-gap/}} Similar nudge unit teams are becoming more popular around the world and have been implemented in other countries such as the United States, Denmark, and Italy~\cite{DelBalzoNudging}. Nudges are useful for automated recommendations to software engineers because they are interventions that are "easy and cheap"~\cite[p.~6]{sunstein2008nudge}. Nudging for software engineers involves allowing alternative options for developers in addition to not providing incentives to encourage the selection of a desired target behavior. In this work, I aim to study the impact of using nudges to improve the decision-making of software engineers when faced with the choice to adopt or ignore useful developer behaviors.


\vspace{-7pt}

\subsubsection{Digital Nudges.}

Digital nudging refers to using technology and user interface design elements to nudge user behaviors in digital choice environments~\cite{weinmann2016digitalnudging}. For example, the FitBit\footnote{\url{https://www.fitbit.com/}} smart watch nudges users to increase physical activity and adopt healthier lifestyle behaviors by monitoring exercise activity, providing feedback to users, and presenting data collected from friends and other users~\cite{weinmann2016digitalnudging}. Weinmann argues understanding digital nudges is becoming increasingly important as more decisions are being made online because the designs of these systems will ``always (either deliberately or accidentally) influences people's choices"~\cite[p.~433]{weinmann2016digitalnudging}. Additionally, Mirsch and colleagues argue that implementing digital nudging provides more advantages because they are ``easier, faster and cheaper" and provide a lot more specified functionality compared to traditional nudges~\cite[p.~635]{mirsch2017digital}. While most prior work examining digital nudges examines their impact on the decision-making of software users, there is limited work exploring how they influence the choices and behavior of software developers. Software engineers are constantly presented with decisions in digital choice environments while writing code, such as whether or not to adopt useful programming behaviors and practices in their work. 

\subsubsection{Choice Architecture.}

 Nudges and digital nudges are useful for improving human behavior because of their ability to influence the context and environment surrounding decision-making, or \textit{choice architecture}~\cite{thaler2014choice}. Thaler and Sunstein note ``nudges are everywhere" and ``choice architecture, both good and bad, is pervasive and unavoidable...Choice architects can preserve freedom of choice while also nudging people in directions that will improve their lives"~\cite[p.~255]{sunstein2008nudge}. Choice architecture is based on the fact that the presentation of choices often impacts decisions made. For example, one specific concept in choice architecture is the ``default rule", which suggests decision-makers are most likely to select default options when making decisions. For example, the Washington State Parks Department modified the default for drivers to opt-out of an optional state park fee and raised over \$1 million to support their state parks.\footnote{\url{http://nudges.org/2009/10/21/switching-the-default-rule-to-save-state-parks-in-washington-state/}} To increase the effectiveness of developer recommendations, I plan to use choice architecture to suggest design implications to automatically present developer behavior recommendations in digital choice environments.

Throughout this proposal, the term \textit{nudge} is used to describe the implementation of digital nudges to improve developer behavior. This includes designing digital choice architectures that suggest beneficial practices to software engineers without providing incentives, restricting options, or forcing actions. Furthermore, while nudge theory can be applied to many different facets of software engineering such as the design of IDEs, programming languages, and physical workspaces, this work primarily focuses on implementing nudges and improving choice architectures for developers while completing programming tasks. The primary developer decision-making environment used for this research is GitHub, a popular online code hosting site with over 31 million developers, 96 million repositories, and 1 billion of code contributions.\footnote{\url{https://octoverse.github.com/}} To create these nudges, I plan to design and evaluate automated recommendations with software robots, or \textit{bots}, to recommend developer behaviors. Thus, this work aims to discover if nudges can encourage developers to adopt useful software engineering practices when faced with choices during real-world programming situations.
\vspace{-5pt}

\subsection{Developer Behaviors}

Developer behaviors refer to practices designed to support and aid software developers in the completion of programming tasks. An example of one of these beneficial developer behaviors is tool adoption. The IEEE Software Engineering Body of Knowledge (SWEBOK), a suite of widely accepted software engineering practices and standards, suggests using development tools is a ``good practice" and can ``enhance the chances of success over a wide range of project”~\cite[p.~A-4]{SWEBOK}. Jazayeri further argues that development tools have become so vital to software engineering that tool usage and the ability to switch between tools should be integrated into software engineering education~\cite{jazayeri2004education}. Researchers and toolsmiths have created tools to help developers save time and effort completing programming tasks and evaluated their impact on development teams and products. For example, static analysis tools are systems which automatically examine code to find and detect errors without running the program. Studies show static analysis tools provide many benefits to projects such as improving code quality~\cite{GoogleFixit}, preventing errors~\cite{bessey2010few}, decreasing debugging time~\cite{Williams2007FaultFixTime}, lowering development costs, and reducing developer effort~\cite{singh2017staticreview}. However, research also shows developers often ignore these tools in practice. Johnson discovered that developers often don't use static analysis tools primarily because of their result understandability, customizability, and false positives in the tool output~\cite{Johnson2013Why}. Similarly, researchers have explored why software engineers avoid using development tools for security~\cite{Xiao2014Security},  debugging~\cite{Cao2010Debugging}, refactoring~\cite{Murphy-HillBarriersRefactoring}, documentation~\cite{Forward2002Documentation}, build automation~\cite{Akond2017BuildTools}, continuous integration~\cite{hilton2017CI}, and more. 

This developer behavior adoption problem also exists for other beneficial software engineering activities outside of development tool usage. For instance, research shows implementing agile software development methodologies provides benefits to teams such as improved communication, faster releases, increased flexibility in design, and improved code quality~\cite{begel2007usage} and the SWEBOK presents agile as a useful software engineering method~\cite[p.~9-9]{SWEBOK}. However, Nerur outlined challenges hindering migration to agile processes that impact the Management and organization, People, Processes, and Technologies preventing teams from adopting this practice~\cite{nerur2005challenges}. The goal of my work is to make effective recommendations to software engineers to encourage adoption of useful developer behaviors. Tilley and colleagues argue that adoption of research-off-the-shelf (ROTS) software developed for industry practitioners should be a primary goal for software engineering researchers~\cite{Tilley2003ROTS}. Furthermore, Wohlin presents general challenges with integrating empirical software engineering research from academia into industry including lack of trust, differing goals, and the transferring of knowledge and technologies~\cite{wohlin2013empirical}. To help bridge the gap between software engineering research and practice, I aim to explore the impact of using nudge theory to increase awareness and encourage adoption of beneficial developer behaviors evaluated by researchers for improving code quality and developer productivity in industry.

% Furthermore, improving adoption of useful developer behaviors can affect the software engineering industry which saw over 3.7 billion users impacted by buggy code and approximately \$1.7 trillion spent on software defects in 2017~\cite{SoftwareFailWatch}. 
 

 %While the design and implementation of software engineering tools and company policies can be barriers preventing developer adoption, this research focuses on the discoverability barrier, where users are unaware that a useful tool exists~\cite{Murphy-HillScreencastingDiscovery}. 

