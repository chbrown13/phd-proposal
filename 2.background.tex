\section{Background}

This section presents background information on key concepts that are important for my research work, namely nudge theory.

\subsection{Nudge Theory}

% What is nudge theory and why use it?
We aim to improve tool automated recommendations by incorporating persuasion and behavioral science concepts from \textit{nudge theory}. A nudge is defined as ``any aspect of choice architecture that alters behavior in a predictable way without forbidding alternatives or significantly changing economic incentives"~\cite[p.~6]{sunstein2008nudge}. Nudges impact how humans make common everyday decisions, such as encouraging people to recycle more by increasing the size of recycling bins\footnote{http://nudges.org/2011/05/02/a-strategy-for-recycling-change-the-recyling-bin-to-garbage-bin-ratio/} and re-labelling trash cans as ``landfills"\footnote{http://nudges.org/2010/08/25/its-not-a-garbage-can-its-a-small-landfill/}. In these cases, people still have the option not to recycle and are not rewarded for recycling, but are still persuaded by the size and naming of bins. Nudges are also used on a much larger scale to impact human behavior and decision-making. For example, the UK government implemented a Behavioural Insights Team\footnote{https://www.gov.uk/government/organisations/behavioural-insights-team}, also known as the Nudge Unit, to improve civic behavior and decisions by citizens. An example of a nudge from this team includes encouraging companies to improve the recruitment of women, promotion of female employees, and reduce the gender pay gap.\footnote{https://www.bi.team/wp-content/uploads/2018/06/GEO_BIT_INSIGHT_A4_WEB.pdf} Similar nudge unit teams are also becoming more popular in other countries around the world~\cite{DelBalzoNudging}.

The study of the context and environment surrounding how people make decisions is known as choice architecture~\cite{thaler2014choice}. There are many aspects of choice architecture that impact human behavior and decision-making, however we focus on \textit{nudge theory}. Nudges are useful for automated recommendations because they are interventions that are ``easy and cheap"~\cite[p.~6]{sunstein2008nudge}. Nudging humans is easy because it does not ban alternative options for choosers, and it's cheap since incentives are not provided to encourage the selection of a target option. Additionally, Sunstein and Thaler outline five situations where humans are more likely to bad good choices and need help making decisions~\cite[p.~74-82]{sunstein2008nudge}. They argue that nudges can be effective for improving human behavior and decision-making in these cases. % We present these scenarios below and describe how they correlate with software engineers choosing whether or not to adopt development tools to show how nudges are applicable for improving tool recommendations:

% \paragraph{Benefits Now--Costs Later.}

% Psychology research shows that humans often participate in activities that provide short term benefits but have long term costs, such as drinking, smoking, and eating junk food. Nudges can be used as a method to prevent these types of behaviors and raise awareness of the consequences for these activities. For instance, many students binge drink and smoke due to perceptions of their peers. However, officials in Montana were able to decrease these activities by nudging students that these ideas were false and alert them that 81\% of students have less than five drinks and 70\% are tobacco free~\cite{linkenbach2003most}. In software engineering, avoiding developer actions may provide some temporary benefits to teams but can ultimately cause significant problems. For example, Xiao and colleagues found that developers often avoid adopting security tools to avoid save time and costs on implementing and training for new systems~\cite{Xiao2014Security}. While saving time to work on prioritized development tasks and saving money to spend other resources are important benefits, ignoring security tools can have more serious consequences in the long run.

% \paragraph{Degree of Difficulty.}

% Humans make approximately 35,000 decisions every day.\footnote{https://go.roberts.edu/leadingedge/the-great-choices-of-strategic-leaders} Nudge theory was developed as a way to improve human decision-making, and research suggests that people are more likely to need and accept help making selections when faced with solving more difficult and complex decisions. Sunstein and Thaler note that ``difficult choices are good candidates for nudges", for example selecting the right mortgage plan compared to choosing the right loaf of bread~\cite[p.~77]{sunstein2008nudge}. One example of how nudges can help is by improving the structure of complex choices. For instance, the Benjamin Moore paints website\footnote{https://www.benjaminmoore.com/en-us/color-overview/color-palettes/color-families} organizes paint colors by similarity (reds, blues, etc.) as opposed to alphabetically (Arctic White, Azure Blue, etc.) which is less helpful to users~\cite[p.~98]{sunstein2008nudge} Software developers are often faced with difficult and challenging choices in their work. The Hacker Noon blog refers to decision-making as ``the most undervalued skill in software engineering" and ``the most important skill in software development", even moreso than coding skills.\footnote{https://hackernoon.com/decision-making-the-most-undervalued-skill-in-software-engineering-f9b8e5835ca6} Furthermore, Li and colleagues interviewed developers and discovered that the ability to make effective decisions is an important attribute in their model of what makes a great software engineer~\cite{GreatSoftwareEngineer}. The tough decisions these software engineers make require knowledge in a variety of topics such as the technical domain, customers and business, tools and building materials, engineering practices, people and organizations, and more. Using digital nudges to make effective recommendations can help improve software engineer decision-making by encouraging them to adopt useful behaviors when faced with difficult choices during the development process.

% \paragraph{Frequency.}

% Difficult problems and decisions become easier with more opportunities. Or in other words, ``practice makes perfect"~\cite[p.~76]{sunstein2008nudge}. Nudge theory argues that nudges are useful for impacting decision-making in these situations. For example, \todo{example} In software engineering, developers are frequently in situations where development tools are useful. \todo{example}. By implementing automated recommendations as nudges, we believe developers are more likely to adopt development tools after discovering frequent opportunities to use them in their work.

% \paragraph{Feedback.} 

% While frequency of choices is also impacts decision-making, learning from previous opportunities is also important. People tend to make poor decisions without feedback ``immediate, clear feedback"~\cite[p.~77]{sunstein2008nudge}. For example, households in San Marcos, CA were nudged to significantly change the amount of energy used after given feedback about their neighbors' energy usage~\cite{schultz2007constructive}. Feedback is also important in software engineering and helping developers. Barik and colleagues found that software developers significantly preferred compiler error messages with an argument structure, in addition to desiring feedback with resolutions to their problems~\cite{barik2018should}. Furthermore, Johnson and colleagues found that poor feedback was one of the main barriers reported by developers preventing static analysis tool adoption~\cite{Johnson2013Why}. Sunstein and Thaler write that providing feedback is the best way to improve human behavior~\cite[p.~92]{sunstein2008nudge}. By implementing automated development tool recommendations as nudges, we aim to provide programmers with productive feedback to help them discover and adopt useful tools into their development workflow.

% \paragraph{Knowing What You Like.} 

% Most of the time, humans are prone to make decisions based on previous experiences with what they already know and like. People rarely select options that are unfamiliar but nudging is useful for introducing new concepts to them. In this case, many software developers are comfortable with the tools or processes without tools they currently use in their work. Murphy-Hill and colleagues observed this when analyzing barriers to peer interactions in the workplace, with developers noting they often work in ``different programming environments" and ``feel that they do not need to discover a new tool because existing tools will do the job"~\cite[p.~76-77]{Murphy-Hill2015HowDoUsers}. Nudging software engineers can help increase awareness and inform developers of new and better tools for completing programming tasks that are better than their usual familiar methods.

\subsection{Digital Nudges}

Furthermore, the use of technology and user-interface design elements to nudge user behavior in digital choice environments is referred to as \textit{digital nudging}~\cite{weinmann2016digitalnudging}. For example, the FitBit\footnote{https://www.fitbit.com/home} smart watch nudges users to increase physical activity and adopt healthier lifestyle behaviors through a digital activity tracker~\cite{weinmann2016digitalnudging}. Digital nudges are becoming more and more important for improving human behavior and decision-making. Weinmann argues that more important decisions will be made in digital choice environments as the overall use of technology increases, and futher suggests that the future of digital nudging should move beyond impacting decisions in user interfaces and focus on influencing real-world behavior~\cite{weinmann2016digitalnudging}. Additionally, Mirsch and colleagues argue that implementing digital nudging provides more advantages because they are ``easier, faster and cheaper" and provide a lot more specified functionality compared to traditional nudges~\cite[p.~635]{mirsch2017digital}.

Software engineers are constantly presented with decisions to make in digital choice environments while programming, such as whether or not to write code ethically. My work aims to accomplish this by persuading software developers to adopt development tools to improve the quality of their applications. Our research goal is, when programmers are faced with real-world development tasks in their work, to digitally nudge developers to apply a useful software engineering tool rather than ignore it.


\subsection{Tool Adoption}

% What is the tool adoption problem?
Our work focuses on persuading developers to adopt useful software engineering tools in their work. We define a tool as any software command or feature that helps users accomplish a task. Similarly, software engineering or development tools refer to tools designed to perform programming tasks or improve software development processes. The IEEE Software Engineering Body of Knowledge (SWEBOK), a suite of widely accepted software engineering practices and standards, suggests using development tools is a ``good practice" and can ``enhance the chances of success over a wide range of project”~\cite[p.~A-4]{SWEBOK}. Researchers and toolsmiths have developed a wide variety of tools to help developers in complete programming tasks such as static code analysis, dynamic code analysis, debugging, refactoring, documentation, version control, build automation, continuous integration, communication, security, productivity, and more. Studies show software engineering tools are useful for improving code quality~\cite{GoogleFixit}, preventing errors~\cite{bessey2010few}, decreasing debugging time~\cite{Williams2007FaultFixTime}, lowering development costs, reducing developer effort and increasing efficiency~\cite{singh2017staticreview}, and providing many more advantages to developers and software products. Jazayeri further argues that adopting development tools has become so vital to the software development process that tool usage and the ability to switch between tools should be integrated into software engineering education~\cite{jazayeri2004education}.

However, despite the fact that research provides evidence applying software engineering tools is useful and worthwhile, developers often ignore them in practice. Researchers have further explored reasons why software engineers avoid using tools for static analysis~\cite{Johnson2013Why}, security~\cite{Xiao2014Security},  debugging~\cite{Cao2010Debugging}, refactoring~\cite{Murphy-HillBarriersRefactoring}, documentation~\cite{Forward2002Documentation}, open source contributing~\cite{mendez2017open}, build automation~\cite{Akond2017BuildTools}, continuous integration~\cite{hilton2017CI}, and more. There are many barriers preventing software engineers from adopting development tools. For instance, Johnson and colleagues found that result understandability was the main reason developers provided for not using static analysis tools~\cite{Johnson2013Why}. While the design and implementation of software engineering tools and company policies can be barriers preventing developer adoption, this research focuses on the discoverability barrier, where users are unaware that a useful tool exists~\cite{Murphy-HillScreencastingDiscovery}. Increasing awareness of useful software engineering tools and persuading software developers to adopt these tools can help improve code quality and developer productivity in industry.

\subsection{Research Overview}

My research will study ways to improve software engineer behavior by integrating concepts from nudge theory in development tool recommendations to increase tool adoption. To study the impact of nudge theory in development tool recommendations, I will adhere to the following plan of work:

\begin{enumerate}
    \item Determine what makes an effective recommendation
    \item Examine existing recommendation tools and strategies
    \item Develop a new recommender system.
\end{enumerate}